\subsection{État du jeu}

\subsection{Configuration initiale du jeu}
\begin{itemize}
        \item 
        Grille modulable selon le niveau (surface différente)
        \item 
        Bonbons de couleurs (jaune rouge vert bleu orange)
        \item 
        Bonbons spéciaux (explosion horizontale/verticale, poisson, bombe multicolore, bombe à retardement, bombe colorante)
        \item 
        bloqueurs (bloque présence bonbon, bloque le bonbon)
        \item 
        Nombre de coups pour réussir / temps
        \item 
        Score
        \item 
        Jauge à remplir pour les étoiles
        \item 
        Objectif à réaliser
\end{itemize}

\subsection{Évolution de l'état du jeu}
\subsubsection{Vérification du déplacement}
Avant de pouvoir échanger deux cases, il faut vérifier plusieurs choses :
\begin{itemize}
	\item
		Le déplacement ne sort pas de la grille
\begin{lstlisting}
caseDestination.x >= 0 && caseDestination.x < tailleGrilleX && caseDestination.y >= 0 && caseDestination.y < tailleGrilleY
\end{lstlisting}
	\item
		Le déplacement ne va pas sur une case non-jouable
\begin{lstlisting}
grille[caseDestination.x][caseDestination.y] != 0 && grille[caseADeplacer.x][caseADeplacer.y] != 0
\end{lstlisting}
	\item
		Les cases sont côtes à côtes (si on clique autre part, ça change la position de la caseADeplacer) :
\begin{lstlisting}
caseADeplacer.x - caseDestination.x == 1 || caseADeplacer.x - caseDestination.x == -1 || ( caseADeplacer.x - caseDestination.x != 1 && caseADeplacer.x - caseDestination.x != -1 && (caseADeplacer.y - caseDestination.y == 1 || caseADeplacer.y - caseDestination.y == -1) )
\end{lstlisting}
	\item
		Le déplacement va créer un alignement de au moins 3 cases identiques : pour cela, on va tout d'abord échanger les deux cases sélectionnées grace à la case temporaire :
\begin{lstlisting}
caseTemporaire = grille[caseDestination.x][caseDestination.y];
grille[caseDestination.x][caseDestination.y] = grille[caseADeplacer.x][caseADeplacer.y];
grille[caseADeplacer.x][caseADeplacer.y] = caseTemporaire;
\end{lstlisting}
		On peut alors commencer la détection des cases à détruire. Si il n'y a aucune case à détruire (lors de la première fois qu'on cherche), il n'y a donc pas de combinaison, on peut alors rééchanger les cases.

\subsubsection{Détection}	
	
Pour pouvoir détecter toutes les cases à détruire, nous utilisons une autre grille de la même taille que celle contenant les éléments, mais cette nouvelle grille contient des booléens. Chaque case est ainsi associée à un booléen. Si la case doit être détruite, le booléens vaudra vrai.

On va ensuite parcourir la première grille ligne par ligne pour marquer tous les alignements horizontaux puis colonne par colonne pour les combinaisons verticales.

Nous avons besoin de quelques nouvelles variables :

\begin{lstlisting}
int couleurTemp; // Couleur temporaire qui va servir pour verifier que plusieurs cases sont identiques
int nbCasesAlignees; // Sert a compter le nombre de cases alignees
\end{lstlisting}

Pour parcourir la grille :

\begin{lstlisting}
for(int j = 0; j < tailleGrilleY; j++)
{
	couleurTemp = grille[0][j]; // La couleur temporaire est egale a la premiere case de la grille
	nbCasesAlignees = 1; // Le nombre de cases alignees est egal a 1 en debut de ligne
	/** Parcours des lignes **/
	for(int i = 0; i < tailleGrilleX; i++)
	{
		if(couleurTemp == grille[i][j]) // Si la couleur temporaire est la meme dans cette case, on incremente nbCasesAlignees
			nbCasesAlignees++;
		else if(nbCasesAlignees >= 3)
		{
			// Si la couleur n'est pas la meme mais qu'on a un alignement, on note dans le deuxieme tableau
			nbCasesAlignees = 1;
			couleurTemp = grille[i][j];
			pasDeCasesADetruire = false;
		}
		else
		{
			nbCasesAlignees = 1;
			couleurTemp = grille[i][j];
		}

		if(i == tailleGrilleX - 1 && nbCasesAlignees >= 3) 
			/* Si on atteind le bord de la grille et qu'on a un alignement de plus de 3,
			on note dans le tableau apres avoir incremente i, pour utiliser la meme fonction de notation dans l'autre tableau. */
	}
} 
\end{lstlisting}

\end{itemize}
